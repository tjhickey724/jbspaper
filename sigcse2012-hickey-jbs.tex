% This is "sig-alternate.tex" V1.9 April 2009
% This file should be compiled with V2.4 of "sig-alternate.cls" April 2009
%
% This example file demonstrates the use of the 'sig-alternate.cls'
% V2.4 LaTeX2e document class file. It is for those submitting
% articles to ACM Conference Proceedings WHO DO NOT WISH TO
% STRICTLY ADHERE TO THE SIGS (PUBS-BOARD-ENDORSED) STYLE.
% The 'sig-alternate.cls' file will produce a similar-looking,
% albeit, 'tighter' paper resulting in, invariably, fewer pages.
%
% ----------------------------------------------------------------------------------------------------------------
% This .tex file (and associated .cls V2.4) produces:
%       1) The Permission Statement
%       2) The Conference (location) Info information
%       3) The Copyright Line with ACM data
%       4) NO page numbers
%
% as against the acm_proc_article-sp.cls file which
% DOES NOT produce 1) thru' 3) above.
%
% Using 'sig-alternate.cls' you have control, however, from within
% the source .tex file, over both the CopyrightYear
% (defaulted to 200X) and the ACM Copyright Data
% (defaulted to X-XXXXX-XX-X/XX/XX).
% e.g.
% \CopyrightYear{2007} will cause 2007 to appear in the copyright line.
% \crdata{0-12345-67-8/90/12} will cause 0-12345-67-8/90/12 to appear in the copyright line.
%
% ---------------------------------------------------------------------------------------------------------------
% This .tex source is an example which *does* use
% the .bib file (from which the .bbl file % is produced).
% REMEMBER HOWEVER: After having produced the .bbl file,
% and prior to final submission, you *NEED* to 'insert'
% your .bbl file into your source .tex file so as to provide
% ONE 'self-contained' source file.
%
% ================= IF YOU HAVE QUESTIONS =======================
% Questions regarding the SIGS styles, SIGS policies and
% procedures, Conferences etc. should be sent to
% Adrienne Griscti (griscti@acm.org)
%
% Technical questions _only_ to
% Gerald Murray (murray@hq.acm.org)
% ===============================================================
%
% For tracking purposes - this is V1.9 - April 2009

\documentclass{sig-alternate}

\begin{document}
%
% --- Author Metadata here ---
\conferenceinfo{SIGCSE}{2012 Raleigh, North Carolina USA}
\CopyrightYear{2012} % Allows default copyright year (20XX) to be over-ridden - IF NEED BE.
%\crdata{0-12345-67-8/90/01}  % Allows default copyright data (0-89791-88-6/97/05) to be over-ridden - IF NEED BE.
% --- End of Author Metadata ---

\title{ Experience with a Web and Mobile Entrepreneurship Bootcamp summer semester
\titlenote{(Produces the permission block, and
copyright information). For use with
SIG-ALTERNATE.CLS. Supported by ACM.}}
\subtitle{[Extended Abstract]
\titlenote{A full version of this paper is available as ...
 }}
%
% You need the command \numberofauthors to handle the 'placement
% and alignment' of the authors beneath the title.
%
% For aesthetic reasons, we recommend 'three authors at a time'
% i.e. three 'name/affiliation blocks' be placed beneath the title.
%
% NOTE: You are NOT restricted in how many 'rows' of
% "name/affiliations" may appear. We just ask that you restrict
% the number of 'columns' to three.
%
% Because of the available 'opening page real-estate'
% we ask you to refrain from putting more than six authors
% (two rows with three columns) beneath the article title.
% More than six makes the first-page appear very cluttered indeed.
%
% Use the \alignauthor commands to handle the names
% and affiliations for an 'aesthetic maximum' of six authors.
% Add names, affiliations, addresses for
% the seventh etc. author(s) as the argument for the
% \additionalauthors command.
% These 'additional authors' will be output/set for you
% without further effort on your part as the last section in
% the body of your article BEFORE References or any Appendices.

\numberofauthors{3} %  in this sample file, there are a *total*
% of EIGHT authors. SIX appear on the 'first-page' (for formatting
% reasons) and the remaining two appear in the \additionalauthors section.
%
\author{
% You can go ahead and credit any number of authors here,
% e.g. one 'row of three' or two rows (consisting of one row of three
% and a second row of one, two or three).
%
% The command \alignauthor (no curly braces needed) should
% precede each author name, affiliation/snail-mail address and
% e-mail address. Additionally, tag each line of
% affiliation/address with \affaddr, and tag the
% e-mail address with \email.
%
% 1st. author
\alignauthor
Timothy Hickey\\
\affaddr{CS Department}\\
\affaddr{MS018}\\
\affaddr{Brandeis University}\\
\affaddr{Waltham, MA 02454}\\
\email{tjhickey@brandeis.edu}
\alignauthor
Pito Salas\\
\affaddr{CS Department}\\
\affaddr{MS018}\\
\affaddr{Brandeis University}\\
\affaddr{Waltham, MA 02454}\\
\email{pitosalas@brandeis.edu}
}

\maketitle
\begin{abstract}
This paper describes three years of experience with an
intensive 3 course summer semester on web and mobile entrepreneurship for second year CS students and beyond.
\end{abstract}

% A category with the (minimum) three required fields
\category[K.3.2]{Computing Milieux}{Computer and Information Science Education }[Computer Science Education]
\category{K.8.0}{Personal Computing}[Games]

\terms{Human Factors, Experimentation, Design}

\keywords{Entrepreneurship, Mobile Applications, Game Design}

\section{Introduction}
In 2010, partly in response to the economic crisis, our university authorized a new summer program that would consist of three co-requisite classes and would provide a semester of residency equivalent to Fall or Spring.  The courses were to be experiential and intensive. We created a Computer Science version of this program which debuted in 2010 and was focused on teaching web and mobile design in the context of entrepreneurship and software engineering to students who had completed a two semester Java programming course.

The program, which we called the COSI-JBS program, had many novel features.

{\bf Three co-requisite classes.} Students completing the program earned credit for three courses. In the 2012 session, the courses were CS152aj: Web Applications, CS152bj: Social Networks, and CS154aj: The JBS Incubator. 

{\bf Minimal prerequisites.} The program was designed for students who had a solid understanding of Java program. The program attracted a mix of students, about half had not taken the Data Structures and Algorithms course. The students were required to teach themselves the fundamentals of Ruby and to complete a programming assignment in Ruby before the summer program began.

{\bf Intensive pedagogy.} The program consists of 10 weeks of classes, with class time from 10-12 and 1-4 every day. The mornings were typically devoted to the two lecture classes CS152aj and C153aj. The afternoons were a combination of invited lectures, workshops, and supervised lab time where students worked in teams to design, build, and launch a web application with a mobile component. Since all students were taking the same courses, the initial part of the summer could focus more on the lecture style courses while the later weeks consists mostly of incubator work where students were completing their projects.

{\bf Faculty Workload} The workload for the faculty was much greater than for a summer school course. The COSI JBS program was co-taught with two faculty, one TA and 10-15 students. This is an expensive program in terms of faculty time.

{\bf Integration of CS, SE, and Business.}  The COSI JBS program offered a successful combination of Computer Science concepts, Software Engineering techniques, and Software Entrepreneurship practices.

{\bf Playing the Whole Game.} The students who completed this course, gained an introduction to the kind of software development that many of them will participate in as graduates if they enter the software industry. The fact that they were spending 10 weeks designing, developing, and launching a product for their CS154aj Incubator course provided strong motivation for the other more academic components of the course.


\section{Pedagogical Structure of the Summer Program}
One of the most distinctive features of the COSI JBS is that it allows a great flexibility in structuring the learning environment, providing an experience which is more like a summer camp or a bootcamp  or startup than typical summer school. We took advantage of this flexibility in a number of ways. Class time was divided into several different types of pedagogica interaction:
\begin{itemize}
\item {\bf Traditional Lectures.} The academic content was covered in traditional lectures where students had assigned reading, the instructor reviewed the material, interspersing the lecture with questions and guiding class discussions. In the beginning of the summer we had lectures every morning, toward the end of the summer several days were devoted entirely to experiential learning. The students earned credit for two full semester courses through these traditional lectures. The particular content varied each of the three summers but always included some aspect of web and/or mobile application design and development.
\item{\bf Incubator Labs and Summer Projects.}  Students also earned a full course credit for an Incubator class which was run as an afternoon lab. In the first week, students were led through a process in which they brainstormed ideas for summer projects and then divided into teams of 2-4 around products that were approved by the faculty. Over the course of the summer the students were introduced to the best practices of collaborative software design using Agile methodology, we discuss this more in detail in the next section. Each team was assigned a client who worked closely with the team to keep it on track.
\item{\bf CEO Speaker Series.} Once or twice a week we brought in a distinguished practitioner, typically a CEO or CTO of a local Software company, who discussed some aspect of Entrepreneurship using their company to illustrate the points. For example, we had speakers on Agile Development, on Startup Financing, on Social Marketing, on Customer Development, on Hiring and Interviewing, etc. This series was viewed by the students as one of the best parts of the class and it also served to help them build their professional networks. These were typically scheduled in the afternoons after a lunch where 2-3 students would be invited to dine with the speaker and instructors before the talk.
\item {\bf Daily Homework.} At the end of each day, the homework assignment for the next morning was discussed. Students typically were required to complete some reading, apply the concepts covered by completing short coding assignments (if appropriate), reflect on the homework through a post in the course blog, screen record a quick review of their code including a demo of it running. Each morning we began with homework cold calls where students would be selected to present their homeworks. Students maintained a google doc containing links to all of their homework components (blog, videos, code in github) and a graduate student reviewed these links and assigned grades every day.
\item{\bf Weekly Homework.} Each Monday a larger programming project was assigned which required students to solve a larger program using the concepts being taught in the course. For example, in the most recent summer we had a sequence of programming assignments based on building a recommender system for movies based on an online open-source movie database with 10 million reviews. The insights and skills they developed working on these problems would be directly applicable to their own incubator projects.
\item{\bf Weekly Project Presentations} At the end of each week, each group was required to give a presentation on their project following guidelines provided by the faculty. The goal of this activity was to give them practice in presenting their work to a public audience and giving them the opportunity to reflect on their progress each week.
\item{\bf Final Product Showcase.} The COSI JBS program concludes with a Product Showcase in which the teams present their products to a panel of 5-10 professionals and an audience of 50-100 students, staff, and faculty.  The professional panel gives comments about the products after each presentation. 
\end{itemize}


\section{ Teaching Software Practices}
One of the goals of the course is to have students learn what its like to work in a startup creating a product. Most of our graduates do go on to work in Software companies and many of those are startups and some of them gain this kind of experience through summer internships.  The difference between a COSI JBS and a summer internship at a startup is that the JBS is designed both as an academic experience with lectures, homework, course credit, quizzes, oral presentations, etc and it gives primary responsibility for the selection, design, development, marketing and launch of the product to the students themselves. 

In "Making Learning Whole," David Perkins makes an argument that the most effective teaching engages the students in the process that professionals follow.  So when, learning Mathematics students students should gain a strong sense of what it is that Mathematicians do. Sometimes the instructor needs to create a simplified version of the experience which still captures most of the key features.  The COSI JBS is precisely that kind of an experience.

One particularly valuable part of the course is that students are given the opportunity to learn  how to rapidly pick up new technologies and teach themselves how to become proficient. In the course we give workshops on the underlying concepts of these technologies but require students to master the details themselves. In the world of professional software development, learning how to seek out and master the latest technologies is a key skill that can be taught. In the last summer, the students were exposed to and mastered the following technologies, none of which they had seen prior to the summer course: 
\begin{itemize}
\item {\bf GIT and github} All programming assignments and code for the team project was hosted on the collaborative editing site, github.com.  Students learned how to branch, merge, resolve conflicts, tag, review, and fork git repositories.
\item {\bf Ruby and Ruby on Rails} The web applications were developed using Ruby on Rails. The students were required to learn Ruby and Ruby on Rails by completing a series of programming projects and using online and print materials to help them master Ruby and the Rails framework.
\item {\bf Heroku} The team projects were all developed using Ruby on Rails and the students learned how to deploy their projects to the web using Heroku which is a site that allows users to deploy Ruby on Rails apps from a git repository and to analyze, debug and tune the deployment remotely using command-line API.
\item {\bf SQL and the RoR ActiveRecord API} We focused on SQL over the summer and students became adept at using the Ruby on Rails ActiveRecord API which models database tables as Ruby objects and provides an API for selecting and modifying records that is independent of the underlying database.  Indeed, they develop locally using MySQL or SQLite and using Postgres when porting to Heroku.
\item {\bf HTML 5, CSS 2.3,  and Twitter Bootstrap} Students learned to specify the layout and styling of webpages using HTML 5 and CSS 2.3 and in particular they learned the Twitter Bootstrap API as well as the SCSS extension of CSS.  
\item {\bf REST, jQuery, and AJAX} Interaction with the webserver was implemented using the REST approach which required an understanding of the HTTP protocol and the HTTP verbs in particular.  jQuery and Ajax were used to facilitate more interactivity in the webpages.
\item {\bf PhoneGap and its Javascript APIs} Mobile application development was presented using the PhoneGap framework.  This approach provides, for each of the standard platforms (Android, iOS, Blackberry, etc.) a base application in which every mobile screen is defined by HTML, CSS, and Javascript.  All interaction between the user interface and the underlying features of the phone (GPS, camera, speakers, microphone, compass, orientation, accelerometer, local storage, etc.) is handled using a Javascript API.  Communication with servers is implemented using Ajax.  The advantage of this approach is that students can, and do, build a full mobile application that interacts with their website using REST protocols in as little as 1-2 weeks.

\item {bf Google Sites, ScreenRecording, Blogging} A less technical, but still quite useful component of the course is learning how to use the google applications (especially google sites and google docs) to create portfolios of their work. They also learn various screen recording tools so they can create videos of code walk throughs to post on their google sites, finally they blog (and comment on blog posts) almost every day which develops written communication skills.
\item {\bf PivotalTracker and other Project management tools}. Students were required to use the Scrum version of Agile methodology in which they specify project goals in the forms of stories which are assigned, weighted, tracked, and classified on a continual basis. There are several different online tools to help support this process and most students used Pivotaltracker.
\item {\bf StackOverflow and Google} One of the key skills students learned over the summer was how to leverage the online presence of fellow coders to solve problems. One of the most effective tools was StackOverflow a Question/Answer site with subdomains devoted to particular technologies (e.g. Ruby on Rails). We have found that many students are reticent to use these online tools, but that after a few required assignments they learned to become both good net citizens (answering questions when appropriate) and good clients (asking appropriate questions only after investing an adequate amount of research into the problem).
\end{itemize}

\section{Curriculum}
Has varied over the three summer programs. Each summer we've
covered some combination of Web and Mobile development. Usually
a course focused on one technology (e.g. Web Design using RoR, or Mobile App Design using Android) and another
on a higher level application domain (e.g. Game Design, or Social Networking),
and a third course based on Product Development and Collaborative Software
Engineering (e.g. the Incubator course). The particular content taught in the two academic courses in the COSI JBS can vary considerably and will likely change every year. What makes the program distinctive is the way in which the content of the two Academic courses is explored more deeply in the Incubator course by applying it to the design, development, and launch of a real product.

\subsection{Fundamental Concepts covered}
Talk about the various concepts we covered in the course, including internet architecture, database fundamentals, recommender systems, SOLID principles, multi-tier hierarchies, security problems and defenses, etc.


\section{Effects of the Program}
The program has been well-received by the students. Our university graduates about 20-25 CS majors each year and the COSI JBS has attracted about 10-15 students each summer, so if this trend continues about half of all CS majors will be taking this course.  

Many students express reservation about the JBS program for a variety of reasons
\begin{itemize}
\item The JBS leaves no time for summer recreation. Students are in class all day and completing homework every evening. Even students in internships will typically have their evenings and weekends off.  
\item Many students use the summer for internships which is not possible with the JBS. A partial answer to this is the extended JBS semester in which students take a full-time internship in the Fall as part of a single long semester (June-Dec) and earn credit for five courses: the three JBS courses, an internship course, and an online readings course in the Fall.  This option only attracts 1-3 students each year so the loss of a traditional summer is a concern for many students.
\end{itemize}

Over the three years that the COSI JBS has been offered as a summer program, we've noticed a considerable increase in entrepreneurial activity among our undergraduates.  About a third of the JBS students continue to build products during the year and two or three have been starting companies each year and seeking funding from accelerators and other startup funding sources.  Its possible that the students who are attracted to the JBS are the ones who would be starting companies anyway, so we don't have proof of causation, but there has been a noticeable change in the entrepreneurial culture among our majors in the past few years.

One challenge with the JBS is that it requires a tremendous amount of faculty time and the need to compensate the faculty makes this an expensive program for the university which brings in much less revenue that summer school or regular courses.

\section{Conclusions and Future Directions}

This novel approach to CS education has been successful on our campus. About half of our majors are projected to participate in the program at some point in their career. The program has been correlated with a perceived uptick in undergraduate entrepreneurial activity, but requires a major investment in faculty time to implement the apprenticeship-style model of teaching.


%ACKNOWLEDGMENTS are optional
\section*{Acknowledgments}
%
% The following two commands are all you need in the
% initial runs of your .tex file to
% produce the bibliography for the citations in your paper.
\bibliographystyle{abbrv}
\bibliography{sigproc}  % sigproc.bib is the name of the Bibliography in this case
% You must have a proper ".bib" file
%  and remember to run:
% latex bibtex latex latex
% to resolve all references
%
% ACM needs 'a single self-contained file'!
%

\section*{References}
Generated by bibtex from your ~.bib file.  Run latex,
then bibtex, then latex twice (to resolve references)
to create the ~.bbl file.  Insert that ~.bbl file into
the .tex source file and comment out
the command \texttt{{\char'134}thebibliography}.
% This next section command marks the start of
% Appendix B, and does not continue the present hierarchy

%\balancecolumns % GM June 2007
% That's all folks!
\end{document}
